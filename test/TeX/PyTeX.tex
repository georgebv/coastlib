\documentclass[12pt, letterpaper]{article}

\usepackage[utf8]{inputenc}
\usepackage[english]{babel}
\usepackage{graphicx}
\usepackage{transparent}
\usepackage{microtype}
\usepackage[dvipsnames]{xcolor}
\usepackage[left=1in, right=1in, top=1.5in, bottom=1in, headsep=1in]{geometry}
\usepackage[none]{hyphenat}
\setlength{\parindent}{0pt}
\setlength{\parskip}{0.2in}
\usepackage{fancyhdr}
\usepackage{mathtools}
\usepackage{multicol}
\usepackage{multirow}
\usepackage{tikz}
\usetikzlibrary{shapes,arrows}
\usepackage{caption}
\usepackage{subcaption}
\usepackage[export]{adjustbox}
\usepackage{pdflscape}
\usepackage{float}

\pagestyle{fancy}
\usepackage{lastpage}
\usepackage{titlesec}
\renewcommand{\baselinestretch}{1.5}
\renewcommand{\headrulewidth}{0pt}
\renewcommand{\footrulewidth}{0pt}
\fancyheadoffset{0.65in}

\begin{document}
							% TITLE SECTION
\begin{multicols}{2}
\hspace*{-0.1in}{\sffamily
	{\normalsize
		\begin{tabular}{ll}
		{\large\textbf{MEMO}}&\\[0.2in]
		{\color{RedOrange}\textbf{TITLE}}& Python / TeX Introduction \\
		{\color{RedOrange}\textbf{DATE}}& \today\\
		{\color{RedOrange}\textbf{TO}}& John Doe\\
		{\color{RedOrange}\textbf{COPY}}& Jane Doe\\
		{\color{RedOrange}\textbf{FROM}}& Me\\
		{\color{RedOrange}\textbf{PROJECT NO}}& A123456\\
		\end{tabular}
	}
}
\hspace*{1.2in}{\sffamily
		{\normalsize
			\begin{tabular}{rl}
			\\[0.2in]
			{\color{RedOrange}\textbf{ADDRESS}}& \footnotesize{COWI North America, Inc.}\\
			{}& \footnotesize{276 5th Avenue}\\
			{}& \footnotesize{Suite 1006}\\
			{}& \footnotesize{New York, NY 10001}\\
			{}& \footnotesize{USA}\\%[0.2in]
			{\color{RedOrange}\textbf{TEL}}& \footnotesize{+1 (646) 545 2125}\\
			{\color{RedOrange}\textbf{WWW}}& \footnotesize{cowi-na.com}\\
			\end{tabular}
		}
	}
\end{multicols}
								% HEADER
\rhead{
\includegraphics[width=2in]{./Images/COWI_logo.png}\\[0.2in]
{{\sffamily\normalsize\color{RedOrange}\textbf{PAGE}} \thepage/\pageref{LastPage}}
}
\lhead{}
								% FOOTER
{\transparent{0.6}\tikz[overlay,remember picture] \node[opacity=0.3, at=(current page.south east),anchor=south east,inner sep=0.2in] {\includegraphics[width=6in]{./Images/origami_memo.png}};}
\cfoot{}
\pagebreak
							% DOCUMENT BODY
\section{My first PyTeX section}
Something very technical goes here\\


We can even have some fancy math!

This is the dispersion relation:
$$\omega^2 = gk\, tanh(kh)$$
where,
$$\omega = \frac{2\pi}{T} = \frac{2\cdot 3.14}{10} = 0.63$$
$$k = \frac{2\pi}{L}$$

Now lets solve it for wave number $k$ with $g=9.81\, m/s^2$ and $h=5\, m$:
$$0.63^2=9.81\cdot k\cdot tanh(k\cdot 5)$$
Python finds the solution with the iterative Newton-Rhapson method, which gives us:
$$k = 0.09$$
which, in turn, gives us wave length:
$$L = \frac{2\pi}{k} = \frac{2\cdot 3.14}{0.09} = 67.67\, m$$

\pagebreak
We can also have figures automatically generated!
\begin{figure}[h]
	\centering
	\includegraphics[height=0.3\textheight]{./Images/sqrt.png}
	\caption{Square root plot}
\end{figure}

\begin{figure}[h]
	\centering
	\includegraphics[height=0.3\textheight]{./Images/hist.png}
	\caption{10000 normally distributed random values histogram plot}
\end{figure}
\end{document}